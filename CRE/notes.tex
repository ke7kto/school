%	File: Notes.tex
%	Created: 7/18/13
%	Last Change: 7/18/13
%
\documentclass[a4paper]{article}
\usepackage[sumlimits,]{amsmath}
\usepackage[]{siunitx}
\usepackage[T1]{fontenc}
\usepackage[ansinew]{inputenc}
\usepackage{showlabels}
\usepackage[margin=1in]{geometry}
\begin{document}

The Design equations are:

\begin{eqnarray}
	\frac{ \partial F_A}{ \partial V_R}&= r_A \label{PFR}\\
	\frac{ \partial N_A}{ \partial t}&= F_{A0}-F_A+r_A V_R \label{CSTR}\\
	\frac{ \partial N_A}{ \partial t}&= r_A V_R \label{Batch}
\end{eqnarray}

In order to find reaction rates $r_A$, you should:
\begin{list}{ }{ }
	\item Create table of t, Concentrations of reactants, and $\frac{ \Delta C}{ \Delta t}$
	\item watch the fit, sometimes you may need to ignore the endpoints.
\end{list}

Alternate method

\begin{list}{}{}
	\item $-\frac{ \partial C_A}{ \partial t}= -r_A$
	\item take the logs of everything
		\begin{eqnarray}
			\ln{-r_A}=\ln{k}+\alpha \ln{C_A}+\beta \ln{C_B}
			\label{log scale of design equation}
		\end{eqnarray}
	\item Do a linear regression of the data, since you will not have a perfect fit
	\item You can also do this with $C_B$ in excess
	\item \begin{eqnarray}
			\ln{-r_A}&=&\ln{k}+\alpha \ln{C_A}+\beta \ln{C_{B0}}\\
			\ln{-r_A}&=&\alpha \ln{C_A}+(\beta \ln{C_{B0}}+ \ln{k})
		\end{eqnarray}
	\item Repeat the experiment with a different $C_B$
	\item You might think to do this by putting $C_A$ in excess. This would mean, however, that you are getting $r_B$, not $r_A$, so $\ln{k_A} \not= \ln{k_B}$ is not true.
\end{list}

\begin{list}{}{}
\item The third way is to get just a little bit of data in an attempt to get the forward reaction rate only. Then change $C_A$ and $C_B$ and run it again.
\item This is a time when you would regress the data directly (using $\ln{}$s).
\end{list}

Polymath nonlinear data regression
\begin{enumerate}
	\item Open polymath
	\item New
	\item Data Regression
	\item click on columns to rename them
	\item Hit ``Nonlinear''
	\item (Name of column) = (otherColumn)^(parameters)
	\item Put in guesses
	\item 
\end{enumerate}
\end{document} 

