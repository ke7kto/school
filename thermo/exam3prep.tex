\documentclass[a4paper]{article}
\usepackage[sumlimits,]{amsmath}
\usepackage[margin=1in]{geometry}
\usepackage[]{siunitx}
\usepackage[]{nth}
\begin{document}
\begin{enumerate}
\item Standard Phase, 298.15 K, 1 bar
\item When a reaction occurs
\item \begin{eqnarray}
        f&=\phi P \\
	\phi&=e^{\frac{G_{R}}{RT}}\\
	G^{R}&=H^{R}-TS^{R}\\
	H^R&=H-H^{IG}\\
	S^R&=S-S^{IG}
	\end{eqnarray}
\item Outline fugacity coefficient of acetone at boiling point \begin{eqnarray}
		f&=\phi^{sat} P^{sat}\\
		\phi^{sat}&=e^{\int_{0}^{P^{sat}}{\frac{Z(T,P)-1}{P}}{dP}}
	\end{eqnarray}
	You would need to know the boiling point of acetone, the vapor pressure of the normal boiling point (1 atm), and some equation of state in order to get Z
\item Outline how to get fugacity coefficient of subcooled (liquid) acetone \{not saturated\}
	
	You would need to use the above correlations to get the fugacity of acetone at it's boiling point, then apply the poynting correction\ldots
	You can also calculate $\phi$ by using the Lee kessler tables\ldots Then you would need reduced temperature, pressure, and acentricity.
\item How would you \textit{quickly estimate}  the fugacity of pure liquids and gases?
	\begin{eqnarray}
		f_{liq}&=P^{sat}(T)\\
		f_{vap}&=P(T)
		\label{6}
	\end{eqnarray}
\item What is the Lewis-Randall rule used for?

	It gives the ideal properties of any mixture, for use in excess calculations.
\item For what real systems would the Lewis-Randall rule be considered accurate?

	ideal solutions.
\item Why do we use residuals? Derive $H^R$ for \nth{2} order virial EOS

	\begin{eqnarray}
		Z&=1+\frac{B}{V}\\
		Z-1&=\frac{BP}{RT}\\
		\frac{G^R}{RT} &=Z-1\\	
		\frac{\partial \frac{G^R}{RT}}{\partial T} &=(\frac{\partial B}{\partial T}\frac{1}{T}-\frac{BP}{RT^2})(-T)\\
		\frac{H^R}{RT} &= -T \left(\frac{ \partial \frac{G^R}{RT}}{\partial T}\right)_{P,x}\\
		\frac{H^R}{RT} &=\frac{P}{R}(\frac{B}{T}-\frac{\partial B}{ \partial T})
	\end{eqnarray}
\item Why is a different G\textsuperscript{R} formula used for a pressure-explicit EOS than for a volume-explicit EOS?

	Because a cubic EOS is unreasonably difficult to solve for the non-explicit EOS term, so it's easier to just define the problem with respect to density or Pressure
\item Why are pressure ratios in compressors and turbines generally 10 or less?

	Because increasing the pressure ratio also increases the temperature, and increasing the pressure ratio above 10 generally leads to overly high temperatures.
\item What stream variable do we generally assume is constant in evaporators, condensors and other heat exchangers?

	Pressure
\item What is the physical or molecular (not mathematical) definition of $\overline{V}_i$? Of $\overline{H}_i$?

	$\overline{V}_i$ is the change in molar volume that component i contributes, or the Volume of pure i if the mixture behaved ideally.

	$\overline{H}_i$ is the change in enthalpy that component i contributes, or the Enthalpy of pure i if the mixture behaved ideally.
\item You know $\mu_1$ and $\mu_2$ for a binary system. How would you obtain $G$ for the mixture?
	\begin{eqnarray}
		\mu_i &\equiv& \overline{G}_i\\
		G&=\Sigma x_i \overline{G}_i =& \Sigma x_i \mu_i
		\label{14}
	\end{eqnarray}

\end{enumerate}

Pre Exam Problems
\begin{enumerate}
		\setcounter{enumi}{8}
\item Pitzer correlation for $H^R$:
	\begin{eqnarray}
		H^R&=H_{0}^{R}+\omega H_{1}^R
		\label{9}
	\end{eqnarray} 
	Remember that you need to start from a 298.15 K and 1 bar standard state to get accurate values.

\item Remember the formula for enthalpy, the full version
	
	$H^{tot}=\sum_{i}^{n} \int_{298.15}^{T}\: Cp_i dT + H_{f_i}^0$
\item \begin{enumerate}
		\item Remember to check your units! $bar$s don't cancel with $\frac{m^3}{mol}$

	\end{enumerate}
\item \begin{enumerate}
		\item Remember that $\frac{G^E}{RT}$ is unitless
		\item Get $H^E$ in terms of $x_1$ only, no $x_2$.
	\end{enumerate}


\item \begin{enumerate}
		\item Watch your units, A is in kJ/mol
		\item Make sure to check your algebra, probably good to do it explicitly
	\end{enumerate}
\end{enumerate}



\end{document} 

