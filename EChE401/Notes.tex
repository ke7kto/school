\documentclass{article}
\usepackage{mhchem}
\begin{document}
\begin{section}{Day 1}
  Introductory course to chemical engineering communications. Part 1.

  Doris -Hyowan?
  Stephanie
  Liu
  Mi chang
  Mung shao wan - Samantha
  Rui Phu -

  Expected to communicate. Grades are incentives for participation. 50\% of the grade is participation

  \begin{list}{$\cdot$}{}
  \item Harihara Baskaran
  \item Office: 111C A.W. Smith
  \end{list}
  Grading Policy\begin{list}{$\cdot$}{}
  \item Class Participation 50\%
  \item Oral Presentation \& Homework 25\%
  \item Written Paper 25\%
  \end{list}
  Homeworks\begin{list}{$\cdot$}{}
  \item Review of Journal Articles
  \item Oral Presentation of Data
  \item Open Source Journals, Journal Authorships, Guidelines vs. Reality
  \item Oral Presentation
  \end{list}
  Scientific Misconduct\begin{list}{$\cdot$}{}
  \item ``Fabrication, falsification and plagiarism, and doesn't include honest error or differences of opinion''
  \item Anything in the above is considered research misconduct
  \item Prohibited by federal (and state) laws
  \item Can lead to civil and criminal penalties
    \begin{list}{$\cdot$}{}
    \item Debarment from funding eligibility
    \item Retraction/Correction of published articles
    \item Recovery of funds
    \item Loss of employment and prison time. Prison time can actually happen
    \end{list}
  \end{list}
  Falsification: Manipulating research materials, equipment or processes, or changing or omitting data or results such that the research is not accurately represented in the research record
  Plagiarism: Use of another person's ideas, processes, results, or words without giving appropriate credit

  Exclusions\begin{list}{$\cdot$}{}
  \item Limited use of identical or nearly identical phrases that are not substantially misleading or o great significance.
  \item Disputes among former collaborators.
  \end{list}
  \begin{list}{NIH/NSF Guidelines}{}
  \item 
  \end{list}
  \begin{list}{Ethics}{}
  \item 
  \end{list}
\end{section}


\begin{section}{Day 2}
To get a general understanding, try a book

\begin{list}{$\cdot$}{Where to search}
\item Journals
\item Subscription Databases - Go here for high-quality information
\item Google Scholar / PubMed are two free databases
\item Wikipedia
\item $\ldots$
\end{list}
\begin{list}{$\cdot$}{Journals in electrochemistry}
\item Journal of Power Sources (\#1 in the field)
\item Journal of the Electrochemical Society
\end{list}

\begin{list}{$\cdot$}{Things I learned about}
\item knovel
\end{list}
\end{section}
\begin{section}{Sections in a journal article}
	\begin{itemize}
		\item Title
			
			Omit waste words in a title. It should e the fewest words that \underline{accurately} describe the content of the paper. Titles are used to generate keywords in word searches.

			examples:
			\begin{enumerate}
				\item Tracking cancer drugs in living cells by htermal profiling of the proteome
				\item Engineering alcohol tolerance in yeast
				\item Altered sterol composition renders yeast thermotolerant
				\item Evidence for direct molecular oxygen production in \ce{CO_2} photodissociation
			\end{enumerate}
			Keyword list: The keyword list provides the opportunity to add keywords
		\item Abstract
			Concisely states the principal objectives and scope of the investigation where these are not obvious from the title.
			Concisely summarizes results and principal observation
		\item Introduction
			\begin{enumerate}
				\item Background
					Pertinent literature-original articles
					Common mistake is to introduce literature without mentioning the key findings of the cited work.
				\item Significance
					What is the need for the study? 
					What is the impact to the field?
				\item Rationale
					Innovation
					Why this work?
			\end{enumerate}
		\item Materials and Methods
			It provides enough information for a skilled person to be able to repeat your study and results.
			- Basis for repitition of work by others
			Materials
			State the make and model of the materials used (including equipment)

			Measurements and Statistical Methods
			Be precise in describing measurements and include errors of measurement.
			How many samples per condition?
			How many runs during analysis? Different from number of samples

			Ordinary statistical methods (e.g. ANOVA, t-test) need not be mentioned in the methods section, but other 'specialized' tests need to be mentioned and references given.
			- In describing results (in Figure legends), the 'ordinary' methods will be mentioned.
		\item Results and Discussion
			Results
			Present your findings
			- Use of figures and tables is critical

			Present the data after analysis in a condensed way
			Extract key trends and describe them
			- Use subtitles in results for individual findings/trends.

			Discussion
			Based on the evidence shown in results, discuss
			\begin{itemize}
				\item New principles established or old reinforced
				\item New generalizations that can be deduced
				\item Comparison with others' findings
				\item Implications of the findings
				\item Limitations of the work
			\end{itemize}
			Address the objectives of the study and the significance of the results
		\item Other:
			Figure legends should be brief but provide enough details to interpret the results without referring to the text
			Tables: When presenting data in tables\ldots
	\end{itemize}

\end{section}
\begin{section}{9 October 2015}
	\begin{itemize}
		\item Uncertainty about the timing, methods and identity of person(s) responsible for reviewing data. Record everything!!
		\item Partial listings of items to be collected
		\item Data collection instruments
	\end{itemize}

\end{section}
\end{document}
